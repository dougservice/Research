%!TeX spellcheck = ru_RU
\documentclass[fleqn,11pt]{article}

\usepackage{amsmath}
\usepackage{amssymb}
\usepackage{amsthm,xpatch}
\usepackage{mathtools}

\usepackage[utf8]{inputenc}
\usepackage[T2A,T1]{fontenc}
\usepackage[main=russian,english]{babel}

\usepackage{placeins}
\usepackage{csquotes}
\usepackage[backend=biber,hyperref=true,backref=true]{biblatex}
\usepackage[unicode]{hyperref}
\addbibresource{../../Common/Bibliography.bib}

%%\usepackage{showframe}
\usepackage{layout}
\oddsidemargin = 10pt
\marginparwidth = 10pt
\textwidth = 450pt
\textheight = 592pt
\voffset = -20pt
\setlength{\parindent}{0pt}
\everymath{\displaystyle}

\newcommand{\exwidth}{0.80\textwidth}

\newcommand{\RNum}[1]{\uppercase\expandafter{\romannumeral #1\relax}}
\newcommand{\MRN}[1]{\text{\RNum{#1}}}
\newcommand{\NOT}[1]{\overline{#1}}
\newcommand{\DNOT}[1]{\overline{\overline{#1}}}

%% Get rid of period in thm environment.
\makeatletter
\xpatchcmd{\@thm}{.}{}{}{}
\makeatother

\theoremstyle{definition}
\newtheorem{definition}{Определение}
\newtheorem{proposition}{Утверждение}

\newenvironment{myproof}[1][\proofname]{%
  \proof[\rm \bf #1]%
}{\endproof}

\newenvironment{tightenum}{
\begin{enumerate}
  \setlength{\itemsep}{1pt}
  \setlength{\parskip}{0pt}
  \setlength{\parsep}{0pt}}{\end{enumerate}
}

\begin{document}
\title{Комбинаторика: заметки}
\author{Дуглас Вильямович Сервис}
\date{Февраль 2020}
\maketitle

\section{Введение}

\begin{definition}[Множество всех подмножеств] ~\\
  Для дюбого множество $M$ существует множество $P(M)$ всех подмножеств $M$:
  \begin{align*}
    x \in P(V) \iff x \subseteq M
  \end{align*}
\end{definition}

\begin{myproof} ~\\
  Для каждого натурального число $k$ из $0$ до $n$, то есть $k \in \{0, 1, 2, \ldots ,n\}$
\end{myproof}

\begin{proposition} ~\\
  Пусть $M$ конечное множество, которое содержит $n$ элементов. Следует что $P(M)$
  содержит $2^n$ элементов.
\end{proposition}


\printbibliography

\clearpage

\section{Добавление}


\end{document}
