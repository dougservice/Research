%!TeX spellcheck = ru_RU
\documentclass[fleqn,11pt]{article}

\usepackage{amsmath}
\usepackage{amssymb}
\usepackage{amsthm}
\usepackage{mathtools}

\usepackage[utf8]{inputenc}
\usepackage[T2A,T1]{fontenc}
\usepackage[main=russian,english]{babel}

\usepackage{csquotes}
\usepackage{enumitem}
\usepackage[unicode]{hyperref}
\usepackage[backend=biber,hyperref=true,backref=true]{biblatex}
\addbibresource{../../Common/Bibliography.bib}

\usepackage{placeins}
%%\usepackage{showframe}
\usepackage{layout}
\oddsidemargin = 10pt
\marginparwidth = 10pt
\textwidth = 450pt
\textheight = 592pt
\voffset = -20pt

\setlength{\parindent}{0pt}

\newcommand{\RNum}[1]{\uppercase\expandafter{\romannumeral #1\relax}}
\newcommand{\MRN}[1]{\text{\RNum{#1}}}
\newcommand{\NOT}[1]{\overline{#1}}
\newcommand{\DNOT}[1]{\overline{\overline{#1}}}
\newcommand{\rtxt}[1]{\textsf{#1}}

\theoremstyle{definition}
\newtheorem{definition}{Определение}

\newenvironment{tightenum}{
\begin{enumerate}
  \setlength{\itemsep}{1pt}
  \setlength{\parskip}{0pt}
  \setlength{\parsep}{0pt}}{\end{enumerate}
}

\newenvironment{tightitem}{
\begin{itemize}
  \setlength{\itemsep}{1pt}
  \setlength{\parskip}{0pt}
  \setlength{\parsep}{0pt}}{\end{itemize}
}

\begin{document}
\title{Заметки по основной математической терминологии}
\author{Дуглас Вильямович Сервис}
\date{Январь 2019}
\maketitle

\section{Введение}
Главная цель этих заметках состоит в том, что представлять терминологию, определения и обозначения нескольких
основных математических предметов (объектов), и в каком способе выражения о этих предметах правильно читается на
русском языке. Главные источники появляются в \hyperref[sec:bib]{списке литературы}.
\nocite{ArosevaEtal1987}
\nocite{Glazunova2004}
\nocite{WikipediaRussian2019}

\section{Множества числа}
\subsection{Натуральные числа}
Если мы определяем натуральные числа как числа, возникающие при подсчёте, например: \textsl{первый, второй, третий,
$\ldots$}, последовательность натуральных числ начитается с единицы.
\begin{align*}
    \mathbb{N^{*}} & \rightleftharpoons \{1,2,3, \ldots \} \quad \footnotemark \\
    & \text{\rtxt{Множество натуральных числ равносильно по определению $1,2,3$ и т.д.}}
\end{align*}
\footnotetext{See ISO 80000-2}
Однако, когда мы определяем натуральные числа как числа, возникающие при обозначении количества предметов, например:
\textsl{нуль предметов, один предмет, два предмета, $\ldots$},  последовательность натуральных числ начинается с нуля.
\begin{align*}
    \mathbb{N} & \rightleftharpoons \{0,1,2, \ldots \}
\end{align*}

\section{Алгебра}

\subsection{Бинарное (двухместное) отношение}

\subsection{Сложение (Прибавление)}
Сложение одна из основных бинарных операций двух аргументов. Выражение $a + b$ (\rtxt{а плюс бэ}) называется суммой
чисел $a$ и $b$. Переменные $a$ и $b$ называется слагаемыми (слагаемое - склоняется как прилагательное)
\[1 + 1 = 2\]
\begin{tightitem}
    \item \rtxt{Один плюс один будет два.}
    \item \rtxt{Один плюс один равно двум.}
    \item \rtxt{К одному я прибавляю (добавляю, складываю) одни.}
    \item \rtxt{К одному прибавить один будет два.}
\end{tightitem}
\[a + b = c\]
\begin{tightitem}
    \item \rtxt{$a$ плюс $b$ будет $c$.}
    \item \rtxt{Сумма $a$ плюс $b$ равна $c$.}
    \item \rtxt{$a$ плюс $b$ равно $c$.}
\end{tightitem}
\[x + y = z\]
\begin{tightitem}
    \item \rtxt{икс плюс игрек будет зэт.}
    \item \rtxt{Сумма икс плюс игрек равен зэт.}
\end{tightitem}

\subsection{Вычисление}

\subsection{Умножение}

\subsection{Деление}

\subsection{Возведение в степень}

\subsection{Извлечение корня}

\FloatBarrier
\subsection{Приоритет}
\begin{figure}
\begin{tabular}{|c|c|} \hline
    Приоритет \footnotesize{(от высочайшего до нижайшего)} & Операция \\ \hline \hline
    $1$ & возведение в степень, извлечение корня \\ \hline
    $2$ & умножение, деление \\ \hline
    $3$ & сложение, вычитание \\ \hline
\end{tabular}
\caption{}
\end{figure}

\FloatBarrier
\section{Геометрия}

\pagebreak
\printbibliography \label{sec:bib}

\section{Ссылки}
\begin{description}[align=right,labelwidth=4cm]
  \item [Натуральное число] {\href{https://ru.wikipedia.org/wiki/%D0%9D%D0%B0%D1%82%D1%83%D1%80%D0%B0%D0%BB%D1%8C%D0%BD%D0%BE%D0%B5_%D1%87%D0%B8%D1%81%D0%BB%D0%BE}{https://ru.wikipedia.org/wiki/\textcyrillic{Натуральное{\_}число}}}
  \item [Сложение]{\href{https://ru.wikipedia.org/wiki/%D0%A1%D0%BB%D0%BE%D0%B6%D0%B5%D0%BD%D0%B8%D0%B5}{https://ru.wikipedia.org/wiki/\textcyrillic{Сложение}}}
\end{description}


\end{document}
