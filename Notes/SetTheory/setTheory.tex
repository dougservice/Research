%!TeX spellcheck = ru_RU
\documentclass[fleqn,11pt]{article}

\usepackage{amsmath}
\usepackage{amssymb}
\usepackage{amsthm}
\usepackage{mathtools}

\usepackage[utf8]{inputenc}
\usepackage[T2A,T1]{fontenc}
\usepackage[main=russian,english]{babel}

\usepackage{csquotes}
\usepackage[unicode]{hyperref}
\usepackage[backend=biber,hyperref=true,backref=true]{biblatex}
\addbibresource{../../Common/Bibliography.bib}

%%\usepackage{showframe}
\usepackage{layout}
\oddsidemargin = 10pt
\marginparwidth = 10pt
\textwidth = 450pt
\textheight = 592pt
\voffset = -20pt

\setlength{\parindent}{0pt}

\newcommand{\RNum}[1]{\uppercase\expandafter{\romannumeral #1\relax}}
\newcommand{\MRN}[1]{\text{\RNum{#1}}}
\newcommand{\NOT}[1]{\overline{#1}}
\newcommand{\DNOT}[1]{\overline{\overline{#1}}}

\theoremstyle{definition}
\newtheorem{definition}{Определение}

\newenvironment{tightenum}{
\begin{enumerate}
  \setlength{\itemsep}{1pt}
  \setlength{\parskip}{0pt}
  \setlength{\parsep}{0pt}}{\end{enumerate}
}

\begin{document}
\title{Теория множества: заметки}
\author{Дуглас Вильямович Сервис}
\date{Октябрь 2018}
\maketitle

\section{Введение}

\subsection{Атомарной формулы и принадлежность}
Дано два переменных $x$, $y$ обозначающие произвольные множества, где $y$ также является
множеством элементов-множеств, содержащее $x$, мы используем следующее обозначение:
$x \in y$, который читается "<икс принадлежит игрек">, "<икс элемент множества игрек">,
"<икс есть элемент игрек">, "<игрек содержит икс в качестве элемента">, или "<икс является элементом игрек">
($\mathsf{ZF}$ система аксиом Цермело-Френкля).

\subsection{YYYY}
$A = \left\{x  \mid \phi(x) \right\}$ - Множество $A$ состоит из всех таких элементов $x$, для
которых истинно $\phi(x)$. Множество $A$ равно множеству, состоящему из всех таких элементов $x$, для которых истинно $\phi(x)$.

\subsection{Связанные и свободные переменные}
Переменная называется связанной, если она стоит под каким-либо квантором, $\forall$ или $\exists$,
и называется свободной, если она не стоит под квантором.

\begin{itemize}
    \item $\coloneqq$ символ для определения обозначений.
    \item $a \subseteq b \coloneqq \forall x (x \in a \rightarrow x \in b)$~\\
    	\textbf{$\mathbf{a}$ содержится в $\mathbf{b}$} ~\\
    	\textbf{$\mathbf{a}$ есть подмножество $\mathbf{b}$}
    \item $a = b \coloneqq (a \subseteq b) \land (b \subseteq a)$
    \item $a \subset b \coloneqq (a \subseteq b) \land (a \neq b)$ ~\\
    	\textbf{$\mathbf{a}$ есть собственное подмножество $\mathbf{b}$}
    \item $a \notin b \coloneqq \; \rceil(a \in b)$
    \item $a \neq b \coloneqq \rceil(a = b)$
    \item $R \coloneqq \left\{X \mid X \notin X \right\}$ называется регулярным.
    \item $S \coloneqq \left\{X \mid X \in X \right\}$ называется сингулярным.
\end{itemize}

\begin{itemize}
    \item $x \notin y$ - \textbf{Икс не элемент множества игрек}.
\end{itemize}

\begin{itemize}
    \item Пара множеств ~\\
	$\left\{a,b\right\} \coloneqq \left\{ x \mid x = a \lor x = b \right\}$
    \item Одноточечное множество (синглетон) ~\\
    	$\left\{a\right\} \coloneqq \left\{a, a \right\}$
    \item Пересечение множеств $a$ и $b$ ~\\
    	$a \cap b \coloneqq \left\{x \mid x \in a \land x \in b\right\}$
    \item Объединение множеств $a$ и $b$ ~\\
			$a \cup b \coloneqq \left\{x \mid x \in a \lor x \in b\right\}$
		\item Разность множеств $a$ и $b$ ~\\
			$a \setminus b \coloneqq \left\{x \mid x \in a \land x \notin b \right\}$
\end{itemize}

\section{Аксиомы}

\begin{definition}[\textbf{Высказывание}] ~\\
\end{definition}

\end{document}
